%***************************************************************************
% XSPIF developer's guide -  devguide.tex
% Chapter : concepts
% Authors: Remy Muller & Vincent Goudard
%***************************************************************************


\chapter{Introduction}

%-------------------------------------------------------------------------
\section{About XPSIF}
\noindent XSPIF is born after a study \cite{pluginarch} of existing audio plugin standards, namely: Steinberg's VST, Apple's AudioUnit, LADSPA, Cakewalk's DXi; as well as objects in modular softwares such as MAX/MSP\footnote{MAX/MSP from Cycling '74: \url{http://www.cycling74.com}}, jMax\footnote{jMax from IRCAM: \url{http://www.ircam.fr}}, PureData\footnote{Pure Data from Miller Puckette: \url{http://www.pure-data.org}}, and EyesWeb\footnote{EyesWeb from Laboratorio di Informatica Musicale: \url{http://www.eyesweb.org}}.
\noindent This study helped deducing a synthetic abstraction from the existing standards, to ease the development of audio plugins, with the underlying guideline: \textit{``Write once, export many.''}\\

\noindent The XML language was chosen to store the data needed to
design an audio plugin. Its syntax is independant from any language,
and platform, and it is highly flexible. The blocks of code are
written in the standard ANSI C\footnote{C++ is also possible, but it
  narrows the scope of the exported standards}
language, which is also platefom independant.\\ 

%-------------------------------------------------------------------------
\section{Semantics}
\noindent To help make things clearer, let us define some terms we will refer to in this manual, to name actors, objects and actions. 

\subsubsection{Peoples:}
\begin{itemize}
\item{A \textbf{XSPIF user}} (or simply ``user'') is a person who aims
  at developping audio plugins, with the XSPIF assistance. This person
  should only have some basic knowledge of the C language, which was
  chosen for writing the callbacks in XSPIF. 

\item{A \textbf{XSPIF developer}} (or simply ``developer'') is a
  person who develops XSPIF to add new features or standards to its
  scope. This person should know C/C++, XML, and Python. 

\item{A \textbf{plugin user}} is a person who uses audio plugins in an
  audio software called `host'. This person can know nothing at all
  about computer programming. 
\end{itemize}

\subsubsection{Objects:}
\begin{itemize}
\item{A \textbf{meta-plugin}}: is a XML spefication which contains the
  ``essence'' of the plugin: i.e. everything needed to describe the
  plugin's features and behaviour. The meta-plugins are stored in files
  using the \verb|.xspif| extension to differentiate them from other
  XML files.
\end{itemize}

\subsubsection{Actions:}
\begin{itemize}
\item{\textbf{Validating}} is the process of examinating the
  meta-plugin's XML file, and ensure that it matches the structure
  defined in the DTD\footnote{Document Type Definition: see
  \ref{chap2_dtd} and the XML language syntax.} 
\item\textbf{{Parsing}} is the process of reading the XML file, and
  transforming it to a structured object called a DOM\footnote{Document Object Model}.
\item{\textbf{Checking}} is the process of examinating the DOM
  elements, and checking if the description makes sense from the XSPIF
  point of view; e.g. that a parameter min value is less than its max
  value\ldots etc. 
\item{\textbf{Translating}}: is the process of writing the C/C++ source files of a given standard, from the elements of the DOM tree.
\end{itemize}


%-------------------------------------------------------------------------
\section{Open Source License}

\noindent XSPIF is Open Source\footnote{\url{http://www.gnu.org/philosophy/free-sw.html}}, 
released under the General Public 
License\footnote{\url{http://www.gnu.org/licenses/gpl.html}}. This means 
that you are free to redistribute it, as long as you do not make a business 
with it. You are aslo free to modify and improve it: may this guide help you in this task.

