%***************************************************************************
% XSPIF user's guide -  userguide.tex
% Chapter : Introduction
% Authors: Remy Muller & Vincent Goudard
%***************************************************************************

\chapter{Introduction}

\noindent From the XSPIF point of view, a plugin is Digital Signal
Processing module aimed to be integrated inside a host application
within a network, a graph or a chain of other modules. Its ``interface''
is fully defined by its audio input and output ports (also called
pins), its controllable parameters and its control output(s) -- when
control data is derived from the audio stream --.\\
\noindent Therefore, plugins generated with XSPIF are mainly meant for
doing effects, analysis or ``continuous synthesis'' -- i.e. as the notion
of events is not modelized yet, MIDI synthetizers are not handled --.\\
\noindent Furthermore, XSPIF relies on the availability of a generic
Graphical User Interface generated automatically by the host
application as a remote for the plugin.\\

\noindent So what should you use XSPIF for?\\
XSPIF lets you develop plugins for multiple standards from the same
code.\\ 
XSPIF can generate plugin templates from the definition of its
``interface''. \\
XSPIF is a fast way to prototype and test new algorithms.\\
XSPIF can help you starting a new project by generating the skeleton
of your plugin and then let you cutomize it to your needs.\\

\noindent In this manual, you should find all the information or the
pointers you would need to quickly start doing plugins with
XSPIF. There is brief description of the ``technologies'' involved in
XSPIF -- i.e. XML and PYTHON -- and the links to the necessary downloads,
followed by a tutorial showing you how to write your first XSPIF
plugin, generate the sources and compile them.\\

\noindent There are 6 supported standards -- VST, AudioUnits,
LADSPA and Max/MSP, PureData and jMax -- but others are planned to be
added. If you want collaborate to XSPIF by adding a format or a new
feature, please refer to the ``XSPIF: developper guide''
\cite{xspif:developerguide} and if you want to have an overview of the
existing plugin standards and their specificities refer to ``Audio
plugin architectures'' \cite{pluginarch}.

