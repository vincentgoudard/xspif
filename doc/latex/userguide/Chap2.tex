%***************************************************************************
% XSPIF user's guide -  userguide.tex
% Chapter : System requirement - Prerequisites
% Authors: Remy Muller & Vincent Goudard
%***************************************************************************

\chapter{System requirement - Prerequisites}

\section{What you should know}
\noindent To fully understand how to use XSPIF, it is necessary to
have a general understanding of programming techniques, especially in
C and of real-time constraints. However, XSPIF has been designed in order to make it the simplest possible to develop audio plugins.\\
\noindent If you want to develop audio plugins, it is also useful to
have basic knowledge in signal processing. There are a number of
websites, where you can find information, and sample code to start
with. \url{http://www.musicdsp.org}.


\section{Python}
\noindent First of all, XSPIF needs python 2.2 or above to be installed
on your machine. Usually it is installed by default on recent linux
distributions. For windows, there is an installer available at:
\verb|http://www.python.org/download/|
and for Mac OSX we recommand to install python through Fink
(\verb|http://fink.sourceforge.net/|) and FinkCommmander (a front-end for
apt-get, similar to synaptic on linux) until there is a real installer
of PyXML for OSX.  
For more explanation refer to \cite{python:site}.

\section{PyXML 0.8.2 or above}
\noindent You will also need PyXML 0.8.2 or above. It is a python
package dedicated to XML handling. You can find it at
\verb|http://pyxml.sourceforge.net/|. For Mac OSX we highly recommend
to install it through FinkCommander which will automatically check all
the dependencies and show if a new version is available. For Windows,
there is also an installer available in the downloads page.

\noindent For more explanation refer to \cite{pyxml:site}.

\section{Compiler/IDE and SDK}
\subsection{Compiler/IDE}
\begin{itemize}
\item \textbf{OSX:} Our reference IDE is Project Builder which uses gcc for
  mac and is easy to use and to customize to run XSPIF to generate new
		sources before rebuilding the plugin.
\item \textbf{Win:} Although some people have succeed in building VST
  plugins with MingW Dev C++ or Borland C++ Builder. We only provide
  examples for Visual C++ as the vstsdk is primarily designed to be
  used with Visual C++. 
\item \textbf{Linux:} The reference compiler we have used is gcc 
\end{itemize}
\subsection{SDKs}
\begin{itemize}
\item \textbf{VST} Available on Steinberg's web site
  \url{http://ygrabit.steinberg.de/users/ygrabit/public_html/vstsdk/OnlineDoc/vstsdk2.3/index.html}\footnote{\cite{vst:sdk} and \cite{vst:developers}} 
\item \textbf{AU} Available on Apple's site
  \url{http://developer.apple.com/audio/} 
\item \textbf{LADSPA} Available on the LADSPA web site
  \url{http://www.ladspa.org}.\footnote{\cite{LADSPA:site}} 
\item \textbf{PD} Available on PureData's web site
  \url{http://www.puredata.org}. 
\item \textbf{Max/Msp} Available on Cycling 74's web site
  \url{http://www.cycling74.com/products/dldoc.html}.
\item \textbf{jMax} The jMax distribution including all the
  source-code is available on Ircam's freesoftware web-site
  \url{http://freesoftware.ircam.fr/}.
\end{itemize}
