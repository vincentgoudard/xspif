%***************************************************************************
% XSPIF user's guide -  userguide.tex
% Chapter : XSPIF Tutorial
% Authors: Remy Muller & Vincent Goudard
%***************************************************************************

\chapter{XSPIF Tutorial}
\label{tutorial}

\section{Writing a XSPIF meta-plugin}

\noindent A meta-plugin is a XML file beginning with a header
specifying its syntax and followed by the various elements described 
in the last section organized in an arborescent structure whose root
is $<plugin>$\\
\noindent This chapter will explain how to write the meta-plugin with
detailed information on each specific elements, their hierarchy and examples of implementation.\\

\subsection{The header}
\noindent Each XSPIF meta-plugin description has to begin with the following header:
\begin{verbatim}
<?xml version="1.0"?>
<!DOCTYPE plugin SYSTEM "xspif.dtd">
\end{verbatim}
The first line specifies which version of XML is used and the second
line informs the parser where to find the the Document Type Definition
(i.e. the syntaxic rules telling how a xspif file has to be written to be
correct from the XML point of view).

\subsection{$<plugin>$}
\label{plugin}
\noindent This is the main tag which is the unique root of the tree.

\begin{itemize}

\item Attributes: 
\begin{description}
\item[$label$] It will be used to name the plugin class or structure and
  shouldn't have spaces in it.
\item[$plugId$] It should be either a four character constant
  into simple quotes, or an 32-bit unsigned integer identifying the
  plugin. It should be unique.
\item[$manufId$] another ID to be sure that the plugin can be uniquely identified.
\item[$maker$] [optionnal] a string describing the plugin maker
\item[$copyright$] [optionnal] a string to specify the copyright if
  needed. It will be used into the generated sources.
\end{description}

\item Elements:
\begin{description}
\item[$<caption>$] [optionnal] A string describing the plugin
\item[$<comment>$] [optionnal] a tag to add global comments to your code
\item[$<code>$] [optionnal] you can declare local variables, macros,
  functions and additionnals includes segments on top of your file
  which \textbf{has to} be inside a \verb|<![CDATA[...]]>| tag so that
  your code is not analyzed by the parser and can be pasted directly into your
  generated source file(s) ``as-is''.  
\item[$<pin>$] [1 or more] A pin is a port through which the audio signal flows. A
  pin is either input or output and can be multi-channels. see below \ref{pin}
\item[$<param>$] [0 or more] A param is a port for the control input, available to
  the plugin user and to the host. see below \ref{param}
\item[$<controlout>$] [0 or more] A controlout is a port for outputting control
  back to the host or to other plugins when possible. see below \ref{controlout}
\item[$<state>$] [0 or more] A state is keeping an internal information necessary
  for the behaviour of the plugin, but not available to the plugin
  user. see below \ref{state}
\item[$<callback>$] [0 or more] There is a finite number of callbacks,
  which implement the behaviour of the plugin, namely:
  \textit{instantiate, deinstantiate, activate, deactivate}, and
  \textit{process}. see below \ref{callback}
\end{description}

\item Example:\\
\noindent This example defines a plugin (the root element of the XML
tree) whose label is \verb|LowPass| with the plugin ID
\verb|'lowp'| and the manufacturer ID \verb|ReMu|. There are
additionnal information about the maker which is r�my muller and the
copyright which is GPL. this plugin contains a caption \verb|Lowpass|
intented to be a human-readable name, some comments to describe the
plugin and a piece of code which includes an additionnal header and
defines a local function which will be used during by the DSP
algorithm. Note that the $<plugin>$ tag is not closed because it will
also contain child elements.

\begin{verbatim}
<plugin  label = "LowPass" plugId="'lowp'" manufId="'ReMu'" 
         maker="Remy Muller" copyright="GPL">

    <caption>Lowpass</caption>
    <comment>A simple lowpass with saturation</comment>
    <code><![CDATA[
    #include <math.h>
    /******************************************/
    /* independent code here */
    static float saturate(float x)
    {
    if(x>0.f)
        x = 2.f*x-x*x;
    else
        x=  2.f*x + x*x;
    return x;
    }
    /******************************************/
    ]]></code>
</plugin>
\end{verbatim}
\end{itemize}


\subsection{$<pin>$}
\label{pin}
\noindent Audio Ports are declared as pins, which means that a single
stereo input is represented by 1 pin with 2 channels and direction =
In.

\begin{itemize}
\item Attributes:
\begin{description}
\item[$label$] It will be used to name the audio buffers
  associated to the pin inside the DSP algorithm.
\item[$dir$] It tells if the pin is an input or an output.
\item[$channels$] It specifies the number of channels inside the pin.
\end{description}

\item Elements:
\begin{description}
\item[$<caption>$] [optionnal] the friendly name of the pin.
\item[$<comment>$] [optionnal] some comments about the pin.
\end{description}

\item Example:\\
 \noindent Here we define a stereo audio pin which direction is an
 input, its label is input which means that the input buffers will be
 known as \verb|input1[]| and \verb|input2[]|. There is also a caption
 which can be used to show the pins name to the final user and some
 comment to document the code.
\begin{verbatim}
<pin label="input" dir="In" channels="2">
    <caption>Stereo Input</caption>
    <comment>This is a stereo input pin</comment>
</pin>
\end{verbatim}
\end{itemize}

\subsection{$<param>$}
\label{param}
\noindent The parameters are the plugin variables that the user can
control. They can be used either directly inside the process or converted
into internal states.

\begin{itemize}
\item Attributes:
\begin{description}
\item[$label$] It will be used to name the parameter. Its value is
  stored inside the plugin class or structure and is always available
  when calling it directly by the label.
\item[$min$] It specifies the minimum value that the parameter can take
  and should be less than max and default.
\item[$max$] It specifies the maximum value that the parameter can
  take and should be strictly superior than min and default.
\item[$default$] It specifies the default value that the parameter can
  take and should be in the range $[min;max]$
\item[$type$] can be float or int.
\item[$mapping$] can be lin for linear or log for logarithmic. Default
  is linear.
\item[$unit$] [optionnal] a string for GUI display.
\item[$noinlet$] [optionnal] can be ``true'' or ``false'', default is
  ``false''. It is only used in modular hosts so that not all parameters
  appear as an input.  
\end{description}

\item Elements:
\begin{description}
\item[$<caption>$] [optionnal] the friendly name of the parameter.
\item[$<comment>$] [optionnal] some comment about this parameter.
\item[$<code>$] [optionnal] each parameter change can trigger a piece of code to update the plugin's states, or ouptput a control value for example.
  cf. \ref{plugin} 
\end{description}

\item Example:\\
\noindent Here we define a cutoff parameter for a filter, in the range [100, 10000] Herz, 
with a default value of 1000 Hz. It has a logarithmic mapping, because it is more 
convenient for a frequency. This parametere has an associated piece of
code which is triggered when the parameter changes to actualize the
internal state \verb|lambda|. We can also note that in this
computation, the parameter is directly known from its label
(\verb|cutoff|) and that it also requires the samplerate which is
accessed with a macro defined in the XSPIF API.
\begin{verbatim}
<param label="cutoff" min="100.0" max="10000.0" default="1000.0" 
       type="float" mapping="log" unit="Hz">
    <caption>Cutoff frequency (Hz)</caption>
    <code><![CDATA[
    // cutoff and samplerate are both in Hertz
    lambda = exp(- cutoff / XSPIF_GET_SAMPLE_RATE()); 
    ]]></code>
    <comment>This is the cutoff frequency of the plugin</comment>
</param>
\end{verbatim}
\end{itemize}

\subsection{$<state>$}
\label{state}
\noindent The states are all the variables stored inside the plugin,
different from the parameters and that are necessary for the DSP
algorithm -- e.g. you can map cutoff and resonance parameters to
internal filter coefficients -- . Note that, as states can be of any
type and to be compatible with both C and C++, they should be
manually allocated with \verb|malloc()| (or \verb|calloc()|) and freed
with \verb|free()|.

\begin{itemize}
\item Attributes:
\begin{description}
\item[$label$] It will be used to name the state. Their value is stored
  inside the plugin class or structure and are always available when
  calling them directly by their label.
\item[$type$]  type can be anything either buit-in types or user defined.
\end{description}

\item Example of two internal states; one of type float, and a pointer 
to a buffer of floats:
\begin{verbatim}
 <state type="float" label="lambda"></state>
 <state type="float *" label="buffer"></state>
\end{verbatim}
\end{itemize}

\subsection{$<controlout>$}
\label{controlout}
\noindent 

\begin{itemize}
\item Attributes:
\begin{description}
\item[$label$] It will be used as a selector used when sending control
  outside of the plugin.
\item[$min$] It specifies the minimum value that the parameter can take
  and should be less than max and default.
\item[$max$] It specifies the maximum value that the parameter can
  take and should be strictly superior than min and default.
\item[$type$] can be float or int
\item[$mapping$] [optionnal] can be lin for linear or log for
  logarithmic. Default is linear.
\end{description}

\item Elements:
\begin{description}
\item[$<caption>$] [optionnal] cf. \ref{plugin} 
\item[$<comment>$] [optionnal] cf. \ref{plugin} 
\end{description}

\item Example:\\
 \noindent A controlout called \verb|env| for outputting the enveloppe
 of an audio signal, the min, max and mapping are used when this is
 necessary to scale the value to put it into a specific range
 (e.g. 0-127 for MIDI CC or 0-1 for normalized parameters\ldots)
\begin{verbatim}
<controlout label="env" min="0.0" max="10000.0" type="float" mapping="log">
    <caption>Enveloppe</caption>
    <comment>Peak amplitude envelope</comment>
</controlout>
\end{verbatim}
\end{itemize}


\subsection{$<callback>$}
\label{callback}
\noindent The callbacks are not mandatory, so that one can generate
templates. On the other side, having the same callback twice or more
isn't allowed as it wouldn't be meaningful.

\begin{description}

\item \textbf{instantiate}
\noindent is the callback where the user should implement the memory
allocation of all structures he will need with \code{malloc} or
\code{calloc}. Initialization of the states can also be done here. 

\item \textbf{deinstantiate}
\noindent is the callback where the user should free the memory allocated for the structures in \textit{instantiate} with \code{free}.

\item \textbf{activate}
\noindent is called every time the plugin user of host switch
on the plugin. The states can be reinitialized if needed (for example, 
by clearing a buffer used in a delay, so that the delay does not ring 
again when the plugin is re-activated). The parameters should not usually
not be reinitialized, though this is not forbidden.
\noindent If a structure depends on the sample rate (e.g. a delay
buffer), it can be the right place to check if it has changed, and in
this case, reallocate the memory for this structure. 

\item \textbf{deactivate}
\noindent is called every time the plugin user of host switch off the
plugin. The states can be reinitialized if needed. The parameters
should usually not be reinitialized, though this is not forbidden. 

\item \textbf{process}
\noindent is the callback where the user should implement the DSP
algorithm. 
\end{description}

\begin{itemize}

\item Attributes:
\begin{description}
\item[$<label>$] the label can be any of instantiate (constructor),
  deinstantiate (destructor), activate (on), deactivate (off),
  process.
\end{description}

\item Elements:
\begin{description}
\item[$<code>$] [optionnal] The code which will be associated to the callback
  defined by \textit{label}. cf. \ref{plugin} 
\end{description}

\item Examples:\\
\noindent Here we define the \emph{process} callback which, as all the
callback, only contains a piece of code. Note that there are 2 macros
from the XSPIF API that are used in this example:
$XSPIF\_GET\_VECTOR\_SIZE()$ to know the size of the current buffer
and $XSPIF\_WRITE\_SAMPLE()$ to write the processed samples to the the
output buffers.
\begin{verbatim}
<callback label="process">
    <code><![CDATA[
    int i = 0;
    for(i=0;i < XSPIF_GET_VECTOR_SIZE();i++)
    {
      // in and out names are derived from the label in the pin declaration
      lp1 = (1.f-lambda)*input1[i] + lambda*lp1;
      lp2 = (1.f-lambda)*input2[i] + lambda*lp2;
      XSPIF_WRITE_SAMPLE(output1, i, saturate(lp1));
      XSPIF_WRITE_SAMPLE(output2, i, saturate(lp2));  
    }
   ]]></code>
</callback>
\end{verbatim}

\begin{verbatim}
<callback label="instantiate">
    <code><![CDATA[
    // Initialize the internal states
    lambda = lp1 = lp2 = 0.;
    ]]></code>
</callback>
\end{verbatim}

\begin{verbatim}
<callback label="deactivate">
    <code><![CDATA[
    // clears memories so that no sound is output
    // when plugin is reactivated with no input
    lp1 = lp2 = 0.;
    ]]></code>
</callback>
\end{verbatim}
\end{itemize}

\subsection{XSPIF API}
\noindent XSPIF API is quite simple and is limited to a few macros
required to have a full abstraction from the plugin standards behind
XSPIF. Their meaning is explained here after, as well as the part of
the code in which they are allowed.
\noindent \emph{None of these macros are available in the code
  elements associated to the $<plugin>$ element} 
\begin{itemize}
\item $XSPIF\_WRITE\_SAMPLE(dest,index,value)$: \\
  It handles automatically adding or replacing process according to the standard.\\
  Availability:  process callback only.\\
  $dest$ has to be a pointer on a float array.\\
  $index$ is the index in  this array.\\
  $value$ is the  value to be written to the  array.
\item $XSPIF\_CONTROLOUT(label,index,value)$: \\
  It handles control outputs, with sample accuracy in the process callback.\\
  Availability: process callback and parameters' attached code.\\
  $label$ is the label given in the corresponding controlout tag and is
  used as selector for sending control.\\
  $index$ if the offset in  samples relative to the current buffer inside
  the process callback or zero if used outside.\\
  $value$ is the value to be sent outside the
  plugin and can be converted automatically to a specific range
  depending on the standard (e.g. 0 to 127 if MIDI CC are used).
\item $XSPIF\_GET\_SAMPLE\_RATE()$: \\
  It returns a 32-bit float value corresponding to the actual sample-rate
  in Hz (44100Hz by default).\\
  Availability: any callback, and in the parameters' attached code.
\item $XSPIF\_GET\_VECTOR\_SIZE()$: \\
  It returns an integer value corresponding to the current vector size
  i.e. the size of the current size of the buffer to process.\\
  Availability: process callback only.
\end{itemize}

\subsection{C vs C++}
\noindent As some of the standards supported by XSPIF are natively
written in the C language and that the C++ syntax is compatible with
C, it has been decided that the code has to be written in C inside the
meta-plugin description. However if you only plan to use standards
written in C++ you still can use C++ inside the meta-plugin as it is
just a matter of copy and paste, but you'll lose the advantage of
quickly porting your algorithm to many standards.

\section{Code generation}
\noindent Now that you have a well writen meta-plugin inside your
xspif file, you'll want to generate the sources for you plugin. For
that purpose, there is a python script located in \verb|xspif/python|
named \verb|xspif.py|. the syntax is:
\begin{center}
\textbf{python xspif.py \textit{standard} [\textit{path}]/\textit{filename}.xspif}
\end{center}
 If we are inside xspif dsitribution, we could type:
\begin{center}
\textbf{python python/xspif.py vst examples/lowpass.xspif}
\end{center}

\noindent Note that you can replace the standard by `\textbf{all}' which will write the sources for all the supported plateforms.

\section{Compilation and integration}

\subsection{Windows}
\subsubsection{VST}
\noindent With visual C++, create a new empty ``win32 dynamic-link
library''project. Add \emph{yourplugin}.vst.cpp (which describes you plugin) and
\emph{yourplugin}.def (which exports the entry point of your
library) and be sure that your include directories point to the vstsdk
headers. if you're experienced with Visual C++, you can also integrate
the XSPIF script as the first step of your building process using
``custom build''.
\subsubsection{Pure Data}
\noindent A makefile is generated automatically by XSPIF for PureData, but you will also need Microsoft Visual C++.\\
\noindent To build the plugin binary, open the windows shell and type \verb|nmake pd_nt| in the folder where the PureData source are. The makefile is assuming some paths for PureData and Visual Studio: edit the Makefile if these paths do not match yours.\\

\subsection{Linux}
\noindent A makefile is generated automatically by XSPIF for LADSPA and PureData.\\
\noindent For LADSPA:
\begin{itemize}
\item \verb|make| builds the dynamic library.
\item \verb|make install| builds the plugin and copy it in the LADSPA directory. Thus the environment variable LADSPA\_PATH should be defined, as recommanded by the Linux Audio Developers.\\
\item \verb|make clean| remove objects, temporary and core files.
\end{itemize}

\noindent For PureData:
\begin{itemize}
\item \verb|make pd_linux| builds the dynamic library.
\item \verb|make clean| remove the plugin binary, and objects, and core files.
\end{itemize}

\subsection{Mac OSX}
\subsubsection{VST and AudioUnits}
\noindent  It is a good choice for VST and AudioUnits to use
Project Builder as IDE\footnote{Integrated Development
  Environment}. Using XSPIF with Project Builder just consist in adding the
script above as the first step of your building process. That way you
can generate new sources each time you want to build your plugin.
Please refer to the ProjectBuilder examples provided with the XSPIF distribution as a
base for your own work.

\subsubsection{PureData}
\noindent For PD, the makefile generated automatically is quite 
straight forward: Open a shell, and type \verb|make pd_darwin|
in the directory where the plugin's source files are.



