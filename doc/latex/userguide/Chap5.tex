%***************************************************************************
% XSPIF user's guide -  userguide.tex
% Chapter : Standards specific considerations
% Authors: Remy Muller & Vincent Goudard
%***************************************************************************

\chapter{Standards specific considerations}

\section{VST}
\noindent With VST 2.3, the only way to output control, is to use MIDI
Continuous Controlers. For that purpose the values have to be
normalized in the range $[0-127]$, and then sent to the host as MIDI
data as defined by by the MMA\footnote{Midi Manufacturer Association
  \url{http://www.midi.org/}}. 

\section{Audio Units}
\noindent Control output is not yet featured in this standard, but
should come soon with Mac OSX 10.3 also called panther. 
Note that AudioUnit provides a set of predefined types -- namely
decibels, hertz, boolean, percent, seconds, phase, cents,
Degrees\ldots-- with the mapping done automatically. However, as XSPIF
can use arbitrary units with an arbitrary mapping, we only implemented
AudioUnit's ``generic parameters'' which only have a range and default
value with a linear mapping. Hence, if you specify a \emph{log}
mapping for a parameter, it will be ignored. Futur improvement to
XSPIF, could use the \emph{unit} -- which can be arbitraty -- attribute to guess the nature of the parameter.


\section{LADSPA}
\noindent In LADSPA, the control outputs undergo the same mechanism as
writing output audio buffers, and it consists in writing a float value
to a memory location which is or can be shared.\\ 
\noindent Thus, unlike VST and PD,  synchronicity is not ensured and
depends on the host management of the different threads. 

\section{PureData, Max/Msp, jMax}
\noindent With Pure data, Max/Msp and jMax, if the parameter's optionnal attribute
\verb|noinlet| is set to \emph{true}, it means that this specific
parameter will not appear as an inlet on the object's layout. It can
be useful if there are many parameters and you want to save space on
the screen to keep some visibility. However, this parameter will still
be controlable by sending the following message to the left-most inlet
\textit{parameter\_name} value. Note that, in pure data, the object
can also receive the message ``print'' which will display some
information about the plugin, in particular, the parameters' names,
and their range and the messages ``on'' and ``off'' which will call
respectively \emph{activate} and \emph{deactivate}.  